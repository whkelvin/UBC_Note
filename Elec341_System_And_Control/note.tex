\documentclass{article}
\usepackage{amsmath}
\usepackage{graphicx}
\usepackage{siunitx}
\usepackage{float}
\usepackage{gensymb}
\usepackage[dvipsnames]{xcolor}
\usepackage{sectsty}
\usepackage{indentfirst}
\usepackage{enumitem}

%\setlength{\parskip}{1em}

\definecolor{color:background}{RGB}{40,40,40}
\definecolor{color:text}{RGB}{230,230,230}

\pagecolor{color:background}
\color{color:text}
\allsectionsfont{\normalfont\sffamily\bfseries}

\title{Elec341 System And Control}
\author{Kelvin Hsu}


\begin{document}
    \sffamily
       \maketitle
       \newpage

    \section*{The Stability of Linear Feedback System}
    \subsection*{Types of Stability}
    \begin{itemize}
        \item Absolute Stability: A closed loop feedback system that is either stable or not stable.
        \item Relative Stability: Given a closed loop system is stable, we can further 
                                  specify the degree of stability.
    \end{itemize}

    \subsection*{Absolute Stability}
    Absolute Stability of a system can be found by determining that all the poles for the transfer function lie in the left-half 
    s-plane, or equivalently, that all the eigenvalues of the system matrix A lie in the left-half s-plane.

    \subsection*{Relative Stability}
    Relative Stability can be determine by examining the relative locations of all the poles or eigenvalues.

    \subsection*{Stable System}
    A stable system is a dynamic system with a bounded response to a bounded input.
    A linear system is stable if and only if the absolute value of its impulse response integrated over 
    an infinite range, is infinite.\par
    Let g(t) be the impulse response of the system.
    \begin{equation*}
        \int_{0}^{\inf} |g(t)| dt = \inf
    \end{equation*}

    Locations of the poles determine whether the system would have an increasing, neutral, or decreasing response for a disturbance input.
    \begin{itemize}
        \item Poles in left-hand portion of s-plane $\rightarrow$ decreasing response
        \item Poles on $j\omega$ axis $\rightarrow$ neutral response
        \item Poles in right-hand portion of s-plane $\rightarrow$ increasing response
    \end{itemize}

    For a feedback system to be stable, all the poles need to have negative real parts.

    For a feedback system with poles on the $j\omega$ axis and left-hand s-plane, the response of the 
    system would be oscillations for a bounded input, unless the input is a sinusoid whose frequency is equal to the 
    magnitude of the $j\omega$ roots. Such a system is called marginally stable since only 
    certain bounded input would cause the system to be unstable.

    For a feedback system with any poles in the right-hand portion, the system would have unbounded output response.

    \subsection*{Methods to Determine Stability of a System}
    \begin{itemize}
        \item The s-plane approach
        \item The frequency plane approach
        \item The time domain approach
    \end{itemize}

    \subsection*{Routh-Hurwitz Criterion}
    A criterion to determine the stability of linear systems.\par
    The characteristic equation can be written in the form,
    \begin{equation*}
        a_{n}s^{n} + a_{n-1}s^{n-1}....+a_{0} = 0
    \end{equation*}
    The coefficients of the characteristic equations are put into an array of the form,
    \begin{center}
        \begin{tabular}{c| c c c}
            $s^{n}$&$a_{n}$&$a_{n-2}$&$a_{n-4}\, ...$\\
            $s^{n-1}$&$a_{n-1}$&$a_{n-3}$&$a_{n-5}\, ...$\\
            $s^{n-2}$&$b_{n-1}$&$b_{n-3}$&$b_{n-5}\, ...$\\
            $s^{n-3}$&$c_{n-1}$&$a_{n-3}$&$a_{n-5}\, ...$\\
            .&.&.&.\\
            .&.&.&.\\
            .&.&.&.\\
            $s^{0}$&$h_{n-1}$&&
        \end{tabular}
    \end{center}
    where,
    \begin{equation*}
        b_{n-1} = \frac{a_{n-1}a_{n-2} - a_{n}a_{n-3}}{a_{n-1}} = \frac{-1}{a_{n-1}}
            \begin{vmatrix}
                a_{n} & a_{n-2}\\
                a_{n-1} & a_{n-3}
            \end{vmatrix}
    \end{equation*}

    \begin{equation*}
        b_{n-3} = \frac{-1}{a_{n-1}}\begin{vmatrix}a_{n}&a_{n-4}\\a_{n-1}&a_{n-5} \end{vmatrix}
     \end{equation*}
    \begin{equation*}
        c_{n-1} = \frac{-1}{b_{n-1}}\begin{vmatrix}a_{n-1}&a_{n-3}\\b_{n-1}&b_{n-3} \end{vmatrix}
    \end{equation*}

    The Routh-Hurwitz criterion states that the number of roots of the characteristic equation q(s) with 
    positive real parts is equal to the number of changes in sign of the first column of the Routh array.

    There are 4 other conditions that need to be considered.
    \begin{enumerate}
        \item No element in the first column is zero
        \item There is a zero in the first column, but some other elements of the row containing the zero in the 
              first column are nonzero.
        \item There is a zero in the first column, and the other elements of the row containing the zero are also zero
        \item Repeated roots of the characteristic equation on the $j\omega$ axis
    \end{enumerate}

    TODO!!!
    
    \section*{PID Control}




\begin{equation*}
    \int_{\text{Beginning of semester}}^{\text{End of semester}} \text{Nathan's Effort} + \text{Dario's Effort} \,\, dt = 0
\end{equation*}


\end{document}